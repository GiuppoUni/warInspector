\documentclass[twocolumn]{article}
\setlength{\columnsep}{20pt}
%\usepackage{url}
%\usepackage{algorithmic}
\usepackage[a4paper]{geometry}
\usepackage{datetime}
\usepackage[font=small,labelfont=it]{caption}
\usepackage{graphicx}
\usepackage{mathpazo} % use palatino
\usepackage[scaled]{helvet} % helvetica
\usepackage{microtype}
\usepackage{amsmath}
\usepackage{subfigure}
\usepackage{hyperref}

% Letterspacing macros
\newcommand{\spacecaps}[1]{\textls[200]{\MakeUppercase{#1}}}
\newcommand{\spacesc}[1]{\textls[50]{\textsc{\MakeLowercase{#1}}}}

\title{\spacecaps{WarMonopoly}\\ \normalsize \spacesc{Visual Analytics Course Sapienza University of Rome} }

\author{Giuseppe Capaldi 1699498\\ Gianmarco Cariggi}
%\date{\today\\\currenttime}
\date{\today}

\begin{document}
\maketitle

\begin{abstract}
%

In this report, we present results and design process from "WarMonopoly", an html local page embedding D3.js to visualize some SIPRI (Stockholm International Peace Research Institute) datasets. The page contains user interactive views as barcharts, choropleth map and arc map due to the geographical nature of the data involved. It is present also a scatter plot result of MDS (Multi-Dimensional scaling), computed in python and visualized on the same page, not to preprocess the data but to add new informations to the user. Python program to compute and visualize in real-time MDS depending on user parameters, is hosted on an "Heroku dyno" (a cloud container to host python codes).


%
\end{abstract}


\section{Introduction}
%
In order to offer a data visualization work that would have been interisting in his results, we chose to think of a dataset that could involve the entire planet and containing useful information. Searching the web SIPRI website stood out cause of the possibility it offers to analyze and visualize datasets regarding military trade (from official reports of NATO and every single country which participates to this initiative) among countries. The dataset contains these information: suppliers and recipients, the type and number of weapon systems ordered and delivered, the years of deliveries and the financial value of the deal (but only in some cases). 


We started from this to think of what visual analytics technique would have fit best our purpose and our data.



\section{Dataset}
%
To have a dataset ready to be visualized easily we needed a well formatted file, a CSV (Comma Separated Values) would have been ideal, but the only format offered for our specific dataset of interest (the army transfer database) was RTF (Rich Text Format), that is better to visualize a text due to the presence of text formatting elements, less for storing and visualize a dataset. So we downloaded from: \url{http://armstrade.sipri.org/armstrade/page/trade_register.php} selecting all suppliers and all recipients from 2000 to 2018 (2019 was not present at the time), all weapon system and from supplier register. So after a quick analyis of its content, some Python scripts helped us to reach our CSV file containing a dataset with this format [Table 1]:
\begin{list}{•}{•}
\item \textbf{Ordered}: the number of items ordered under the deal
\item \textbf{Weapon model}: the designation of the weapon system concerned
\item \textbf{Weapon category}: description of the weapon system concerned
\item \textbf{Ordered year}: the year the order was placed or, in the case of licensed production, the licence was issued
\item \textbf{Delivered year}: the year or years during which deliveries took place. If no deliveries have yet been made, this field is left blank.
\item \textbf{Delivered num}: the number of items delivered or produced under the deal
\item \textbf{Comments}: any additional information that is known about the deal. This can include the financial value of the deal, what the weapons will ostensibly be used for, whether the weapons are being donated as military aid, and any information on offsets linked to the deal.
 
\item 
\item
\end{list}



In a second phase in order to draw arrows from a supplier country towards the Recipients we joined the upper dataset with a second one containing only latitude-longitude coordinates and country code for both Supplier and Recipient. This new attributes have been used exclusively for visualization purpose.

In order to apply a dimensionality reduction technique, comparing different distance measures and analyzing useful results we used a second dataset [Table 2] from SIPRI, that is obtained using a trend-indicator value (TIV) in order to have for each year an integer number representing how many weapons a country exported. This value takes into account several factors in order to have a consistent measure to compare trades different for number of units, type of weapon ecc.

\onecolumn
\begin{center}

\begin{table}[hbt!]
\resizebox{\textwidth}{!}{
\begin{tabular}{||c|c|c|c|c|c|c|c|c||}
\hline
Supplier & Recipient & Ordered & Weapon model & Weapon category & Ordered year & Delivered year & Delivered num. & Comments             \\
\hline
...         &              &         &              &                 &              &                &                &                      \\
\hline
         Albania  & Burkina Faso & (12)    & M-43 120mm   & Mortar          & (2011)       & 2011           & 12             & Probably second-hand \\
\hline
...         &              &         &              &                 &              &                &                &             \\ 
\hline       
\end{tabular}}
\end{table}
\label{tab1}Table 1

\end{center}

\begin{center}
\begin{table}[hbt!]
\resizebox{\textwidth}{!}{
\begin{tabular}{||l|l|l|l|l|l|l|l|l|l|l|l|l|l|l|l|l|l|l|l|l||}
\hline
          & 2000 & 2001 & 2002 & 2003 & 2004 & 2005 & 2006 & 2007 & 2008 & 2009 & 2010 & 2011 & 2012 & 2013 & 2014 & 2015 & 2016 & 2017 & 2018 & 2019 \\
          \hline
...       &      &      &      &      &      &      &      &      &      &      &      &      &      &      &      &      &      &      &      &      \\
\hline
Australia &      & 43   & 47   & 44   & 2    & 49   & 14   & 18   & 26   & 80   & 115  & 143  & 45   & 54   & 97   & 87   & 134  & 98   & 38   & 148  \\
\hline
...       &      &      &      &      &      &      &      &      &      &      &      &      &      &      &      &      &      &      &      &     \\
\hline
\end{tabular}
}
\end{table}
\label{Table 2} Table 2
\end{center}
\twocolumn

\section{Design process}
%

We started from the general layout of the website using a framework called azle.js to easily divide the page into different panels that will later contain the views.
Due to the nature of the data, that involve countries from all around the world and time series data, we chose to start from a map visualization.
From related works on the field we though that a directed graph having countries as nodes could be less user friendly than a map showing the user exactly from where and to where the expeditions went. In that way user can see clearly expeditions that involved countries very far away. In future implementations this map could be used to let user explore the single transactions, to analyze them one-by-one just after clicking on the arrow corresponding to the related supplier and recipient.

The arc map shows which country traded with, in a user defined range of years between 1980-2018, but this doesn’t give any information about the amount of weapons or the monetary value involved in the transaction. So we chose to introduce a second world map, that is this time a choropleth map (from Greek χῶρος "area/region" and πλῆθος "multitude") that is a type of thematic map in which areas are shaded or patterned in proportion to a variable that represents an aggregate summary of a geographic characteristic within each area, such as population density or per-capita income. In this case selecting a country, the other countries with which the selected country traded with, always in the same user defined range of years of before, are colored in proportion to the amount of units delivered  to them.
This was a critical part of our work in which we tried to find economic data to merge with our model, in order to relate economic value to traded weapons, and do more reasonable comparisons. The website of SIPRI talks about an economic value associated to the transaction but we discovered it is not present in each entry. Despite that the already mentioned TIV value could have been useful for this purpose in this particular dataset SIPRI didn’t included it at all. So with an expert in the field and some additional data, or with direct access to TIV values, a linear combination could solve the problem in the future.
Then we chose to add also a barchart in the bottom panel to ease the comparison between data. Here new information are presented, for the supplier country selected by the user we chose to plot, the total number of ordered weapons on the y axis, and the year in which they have been ordered on the x axis (always in the range specified by the user).

To select supplier country and range of years we though first at slider but then this kind of input was left only to choose range of years and buttons were introduced where each button is the flag of the corresponding country. 

Finally we chose to add MDS analysis using a the already mentioned second dataset and using three different distance measures (1)(2)(3) to compose the dissimilarity matrix.

\begin{align}
    distance_{i,j}^{[1]} &= | d_i-d_j | \label{eq:distance 1}
\end{align}
\begin{align}
    distance_{i,j}^{[2]} &= \frac{|d_i-d_j|}{(d_i+d_j)}\label{eq:distance 1}
\end{align}
\begin{align}
    distance_{i,j}^{[3]} &= \frac{|d_i-d_j|}{(d_i+d_j)/2} \label{eq:distance 1}
\end{align}


\section{Visualization}
\subsection{Arc map}
Arc map shows a world map using Mercator projection object to draw states and keep proportions when a zoom or a drag on the map is applied. We chose to use an high contrast palette to make easier to see yellow arrows connecting the different states.  
\section{Interaction}


\begin{align}
    r &= e \cdot \Delta_t \label{eq:abc}
\end{align}


\section{Results}


\begin{figure}[hbt!]
\centering
    \subfigure[]{\includegraphics[width=.3\linewidth]{fig/black.jpg}\label{fig:demo-standard}}
    \subfigure[]{\includegraphics[width=.3\linewidth]{fig/black.jpg}\label{fig:demo-fancy}}
    \caption{Result from task X showing \subref{fig:demo-standard} standard rendering compared to fancy rendering in \subref{fig:demo-fancy}. Notice the improved smoothness of the fancy rendering which was achieved through the use of Equation~\ref{eq:abc} and \emph{knowledge}. Shading is typically a trade-off between computation time and visual quality.}
    \label{fig:demo}
\end{figure}


\section{Conclusion}


\onecolumn{
\begin{thebibliography}{9}
\bibitem{latexcompanion} 
Michel Goossens, Frank Mittelbach, and Alexander Samarin. 
\textit{The \LaTeX\ Companion}. 
Addison-Wesley, Reading, Massachusetts, 1993.

\bibitem{einstein} 
Albert Einstein. 
\textit{Zur Elektrodynamik bewegter K{\"o}rper}. (German) 
[\textit{On the electrodynamics of moving bodies}]. 
Annalen der Physik, 322(10):891–921, 1905.

\bibitem{knuthwebsite} 
Knuth: Computers and Typesetting,
\\\texttt{http://www-cs-faculty.stanford.edu/\~{}uno/abcde.html}
\end{thebibliography}
}

\end{document}

